\item \points{18} {\bf Constructing kernels}

In class, we saw that by choosing a kernel $K(x,z) = \phi(x)^T\phi(z)$, we can
implicitly map data to a high dimensional space, and have a learning algorithm (e.g SVM or logistic regression)
work in that space. One way to generate kernels is to explicitly define the
mapping $\phi$ to a higher dimensional space, and then work out the
corresponding $K$.

However in this question we are interested in direct construction of kernels.
I.e., suppose we have a function $K(x,z)$ that we think gives an appropriate
similarity measure for our learning problem, and we are considering plugging
$K$ into the SVM as the kernel function. However for $K(x,z)$ to be a valid
kernel, it must correspond to an inner product in some higher dimensional space
resulting from some feature mapping $\phi$.  Mercer's theorem tells us that
$K(x,z)$ is a (Mercer) kernel if and only if for any finite set $\{x^{(1)},
\ldots, x^{(\nexp)}\}$, the square matrix $K \in \Re^{\nexp \times \nexp}$ whose entries
are given by $K_{ij} = K(x^{(i)},x^{(j)})$ is symmetric and positive
semidefinite. You can find more details about Mercer's theorem in the notes,
though the description above is sufficient for this problem.
%
In this question we are interested to see which operations preserve the validity of kernels. 
%

Let $K_1$, $K_2$ be kernels over $\Re^{\di} \times
\Re^{\di}$, let $a \in \Re^+$ be a positive real number, let $f : \Re^{\di} \mapsto
\Re$ be a real-valued function, let $\phi: \Re^{\di} \rightarrow \Re^\nf$ be a
function mapping from $\Re^{\di}$ to $\Re^\nf$, let $K_3$ be a kernel over $\Re^\nf
\times \Re^\nf$, and let $p(x)$ a polynomial over $x$ with \emph{positive}
coefficients.

For each of the functions $K$ below, state whether it is necessarily a
kernel.  If you think it is, prove it; if you think it isn't, give a
counter-example.

\begin{enumerate}

\item \subquestionpoints{1} $K(x,z) = K_1(x,z) + K_2(x,z)$
\item \subquestionpoints{1} $K(x,z) = K_1(x,z) - K_2(x,z)$
\item \subquestionpoints{1} $K(x,z) = a K_1(x,z)$
\item \subquestionpoints{1} $K(x,z) = -a K_1(x,z)$
\item \subquestionpoints{5} $K(x,z) = K_1(x,z)K_2(x,z)$
\item \subquestionpoints{3} $K(x,z) = f(x)f(z)$
\item \subquestionpoints{3} $K(x,z) = K_3(\phi(x),\phi(z))$
\item \subquestionpoints{3} $K(x,z) = p(K_1(x,z))$

\end{enumerate}

[\textbf{Hint:} For part (e), the answer is that $K$ \emph{is} indeed
a kernel. You still have to prove it, though.  (This one may be harder than the
rest.)  This result may also be useful for another part of the problem.]

\ifnum\solutions=1 {
  \begin{answer}
  All 8 cases of proposed kernels $K$ are trivially symmetric because
  $K_1, K_2, K_3$ are symmetric; and because the product of 2 real
  numbers is commutative (for (1f)).  Thanks to Mercer's theorem, it
  is sufficient to prove the corresponding properties for positive
  semidefinite matrices.  To differentiate between matrix and kernel
  function, we'll use $G_i$ to denote a kernel matrix (Gram matrix)
  corresponding to a kernel function $K_i$.

  \begin{enumerate}

  \item Kernel.  The sum of 2 positive semidefinite matrices is a
    positive semidefinite matrix: $\forall z\; z^T G_1 z \geq 0, z^T
    G_2 z \geq 0$ since $K_1, K_2$ are kernels.  This implies $\forall
    z\; z^T G z = z^T G_1 z + z^T G_2 z \geq 0$.

  \item Not a kernel.  Counterexample: Let $K_1$ be a kernel with positive definite matrix $G_1$ and let $K_2 = 2K_1$ (we are using
    (1c) here to claim $K_2$ is a kernel).  Then we have $\forall z\;
    z^T G z = z^T (G_1 - 2G_1) z = -z^T G_1 z < 0$.

  \item Kernel.  $\forall z\; z^T G_1 z \geq 0$, which implies
    $\forall z\; a z^T G_1 z \geq 0$.

  \item Not a kernel.  Counterexample: Let $K_1$ be a kernel with positive definite matrix $G_1$ and set $a = 1$.  Then we have $\forall
    z\; -z^T G_1 z \leq 0$.

  \item Kernel.  $K_1$ is a kernel, thus $\exists \phi^{(1)}$ such that
    $K_1(x, z) = \phi^{(1)} (x)^T \phi^{(1)} (z) = \sum_i \phi_i^{(1)} (x)
    \phi_i^{(1)} (z)$.  Similarly, $K_2$ is a kernel, thus $\exists
    \phi^{(2)}\; K_2(x, z) = \phi^{(2)} (x)^T \phi^{(2)} (z) = \sum_j
    \phi_j^{(2)} (x) \phi_j^{(2)} (z)$.
    \begin{align}
      K(x, z)
      &= K_1(x, z)K_2(x, z) \\
      &= \sum_i \phi_i^{(1)} (x) \phi_i^{(1)} (z) \sum_i \phi_i^{(2)}
      (x) \phi_i^{(2)} (z) \\
      &= \sum_i \sum_j \phi_i^{(1)} (x) \phi_i^{(1)} (z) \phi_j^{(2)}
      (x) \phi_j^{(2)} (z) \\
      &= \sum_i \sum_j (\phi_i^{(1)} (x) \phi_j^{(2)} (x))
      (\phi_i^{(1)} (z) \phi_j^{(2)} (z)) \\
      &= \sum_{(i,j)} \psi_{i, j}(x) \psi_{i, j}(z)
    \end{align}
    Where the last equality holds because that's how we define
    $\psi$.  We see $K$ can be written in the form $K(x, z) =
    \psi(x)^T \psi(z)$ so it is a kernel.

%    Here is an alternate super-slick linear-algebraic proof.  If $G$ is
%    the Gram matrix for the product $K_1 \times K_2$, then $G$ is a
%    principal submatrix of the Kronecker product $G_1 \otimes G_2$, where
%    $G_i$ is the Gram matrix for $K_i$. As the Kronecker product is
%    positive semi-definite, so are its principal submatrices.

  \item Kernel. Just let $\phi(x) = f(x)$, and since $f(x)$ is a
    scalar, we have $K(x, z) = \phi(x)^T \phi(z)$ and we are done.

  \item Kernel. Since $K_3$ is a kernel, the matrix $G_3$ obtained
    for \emph{any} finite set $\{x^{(1)},\ldots,x^{(\nexp)}\}$ is positive
    semidefinite, and so it is also positive semidefinite for the sets
    $\{\phi(x^{(1)}),\ldots,\phi(x^{(\nexp)})\}$.

  \item Kernel. By combining (1a) sum, (1c) scalar product, (1e)
    powers, (1f) constant term, we see that any polynomial of a kernel
    $K_1$ will again be a kernel.

  \end{enumerate}

\end{answer}

} \fi
